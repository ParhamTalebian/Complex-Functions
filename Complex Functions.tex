\documentclass[12pt]{report}
\usepackage{lipsum} % For dummy text, you can remove this line
\usepackage{multicol}
\usepackage{hyperref}
\usepackage{shapepar}
\usepackage{amsmath}
\usepackage{amssymb}
\usepackage{amsfonts}
\usepackage{xecolor}
\usepackage{mathtools}
\usepackage{upgreek}
\usepackage{pgfplots}
\usepgfplotslibrary{fillbetween}
\usepackage{xepersian}
\settextfont{XB Kayhan}

\setlength{\parindent}{10pt} 


\begin{document}
	
	
	\title{توابع مختلط دکتر رضی}
	\author{پرهام طالبیان}
	\date{\today}
	\maketitle
	
	

	\tableofcontents
	\chapter{یادآوری}
	
	\section{اعداد مختلط}
	فرض کنید
	$$\mathbb{R} = \mathbb{R} \times \mathbb{R} = \{(x, y), x, y \in \mathbb{R}\}$$
	حاصلضرب دکارتی اعداد حقیقی باشد در اینصورت مجموعه اعداد مختلط که با
	$\mathbb{C}$
	نمایش داده می‌شود عبارت از 
	$\mathbb{R}^2$
	به همراه اعمال جبری زیر:
	\begin{enumerate}
		\item 
		عمل جمع
		$$(x_1,y_1) + (x_2,y_2) = (x_1+x_2, y_1+y_2)$$
		\item
		عمل ضرب
		$$(x_1,y_1) \times (x_2,y_2) = (x_1x_2 - y_1 y_2, x_1y_2+y_1x_2)$$
		\item
		عمل ضرب اسکالر
		$$\alpha(x, y) = (\alpha x , \alpha y) \,,\, \alpha \in \mathbb{R}$$

	\end{enumerate}
	
	$$\mathbb{C} = \{x+iy, x, y \in \mathbb{R} ; i^2 = -1\}$$
	\subsection{ویژگی ها}
	\begin{enumerate}
		\item 
		هر دو عدد مختلط یک زوج مرتب 
		$z=(x, y)$
		می‌باشد که
		$$x = Re(z) \qquad y = Im(z)$$
		\item
		برای هر عدد مختلط
		$z = (x, y)$
		قدرمطلق z به صورت زیر تعریف می‌شود
		$$|z| = \sqrt{x^2+y^2}$$
		\item
		مزدوج عدد مختلط 
		$z=(x, y)$
		،
		$\bar{z}$
		به صورت زیر تعریف می‌شود
		$$\bar{z} =(x, -y)$$
		\item
		برای هر عدد مختلط  
		$z$
		داریم:
		$z \bar{z} = |z|^2\qquad$
		\item
		قسمت حقیقی و موهومی عدد مختلط
		$z = x+iy$
		بر حسب 
		$z$
		و
		$\bar{z}$
		به صورت زیر است
		$$x=\frac{z+\bar{z}}{2} \qquad y=\frac{z - \bar{z}}{2i}$$
		\item
		تقسیم
		$\frac{z_1}{z_2}$
		به صورت زیر است
		$$\frac{z_1}{z_2} = \frac{z_1\bar{z_2}}{z_2\bar{z_2}} = \frac{z_1\bar{z_2}}{|z_2|^2} = (\frac{x_1x_2+y_1y_2}{x_2^2+y_2^2}, \frac{x_2y_1 - x_1y_2}{x_1^2+x_2^2})$$
	\end{enumerate}
	
	بین اعداد مختلط و زیر مجموعه ماتریس ها تناظر یک به یک به صورت زیر برقرار است.
	
	\[
	a + ib \Leftrightarrow \begin{pmatrix}
		a & -b \\
		b & a \\
		
	\end{pmatrix}
	\]
	اگر 
	$a + ib \neq 0$
	آن‌گاه	
	\[
	a + ib \Leftrightarrow \begin{pmatrix}
		a & -b \\
		b & a \\
		
	\end{pmatrix} \Leftrightarrow
	\sqrt{a^2+b^2} \begin{pmatrix}
		\frac{a}{\sqrt{a^2+b^2}} & \frac{-b}{\sqrt{a^2+b^2}} \\
		\frac{b}{\sqrt{a^2+b^2}} & \frac{a}{\sqrt{a^2+b^2}} \\
	\end{pmatrix}
	\]
	\[
	= \sqrt{a^2+b^2} \begin{pmatrix}
		\cos \phi & -\sin \phi \\
		\sin \phi & \cos \phi 
	\end{pmatrix} \Leftrightarrow |a + ib| \exp^{i \phi}
	\]
	که شامل یک دوران با اندازه 
	$\phi$
	حول مبدا و یک تجانس با ضریب
	$\sqrt{a^2+b^2}$
	می‌باشد.
	
	همچنین برای ضرب در عدد مختلط هم ارزی های زیر را داریم:
	\[
	(a+ib) \times (x + iy) \Leftrightarrow \begin{pmatrix}
		a & -b\\
		b & a\\
	\end{pmatrix} \times \begin{pmatrix}
	x \\
	y\\
\end{pmatrix} \Leftrightarrow\begin{pmatrix}
a & -b\\
b & a\\
\end{pmatrix} \times \begin{pmatrix}
x & -y\\
y & x \\
\end{pmatrix}
	\]
	
	\textbf{مثال:}
	فرض کنید
	$k>0$
	،
	$z_1$
	و
	$z_2$
	اعداد مختلط ثابتی باشند. مکان هندسی
	$|\frac{z - z_1}{z - z_2}| = k$
	را مشخص کنید.
	\[
	|\frac{z - z_1}{z - z_2}|^2 = k^2 \Leftrightarrow (\frac{z-z_1}{z- z_2})(\frac{\bar{z} - \bar{z_1}}{\bar{z} - \bar{z_2}} ) = k^2
	\]
	\[
	\Leftrightarrow (k^2 - 1)z\bar{z} + (z_1 - k^2 z_2)\bar{z} + (\bar{z_1}- k^2 \bar{z_2})z - z_1 \bar{z_1} + k^2 z_2 \bar{z_2}= 0
	\]
	اگر 
	$k \neq 1$
	آنگاه معادله فوق معادله یک دایره است که توسط رابطه زیر مشخص می‌شود
	\[
	|z -\frac{z_1 - k^2z_2}{1 - k^2}| = \frac{k}{|1 - k^2|} |z_1 - z_2| \,,\, z_1 \neq z_2
	\]
	اگر 
	$k = 1$
	\[
	(z_1 - z_2)\bar{z} + (\bar{z_1} - \bar{z_2}) z - z_1\bar{z_1} + z_2\bar{z_2} = 0 \Leftrightarrow |z-z_1| = |z-z_2|
	\]
	که معادله عمود منصف خطی است که 
	$z_1$
	را به 
	$z_2$
	وصل می‌کند.
	
	\textbf{نکته:}
	دو بردار 
	$z_1$
	و
	$z_2$
	را موازی گویند هر گاه عدد حقیقی غیر صفر 
	$k$
	وجود داشته باشد بطوریکه
	$$z_1 = kz_2 \Leftrightarrow z_1 \bar{z_2} = k |z_2|^2 \Rightarrow Im\{z_1 \bar{z_2}\}= 0$$
	دو بردار 
	$z_1$
	و 
	$z_2$
	عمود بر هم گویند اگر و تنها اگر عدد حقیقی غیر صفر 
	$k$
	وجود داشته باشد بطوریکه
	$z_1 = k z_2 e^{i\frac{\pi}{2}}$
	
	\subsection{شناسه یا آرگومان}
	اندازه ای از زاویه
	$\theta$
	که بردار غیر صفر
	$z$
	با محور حقیقی مثبت می‌سازد یک شناسه یا آرگومان نامیده می‌شود و با
	$\arg{z}$
	نمایش داده می‌شود
	$$\cos (\arg z) = \frac{Re\{z\}}{|z|} \qquad \sin(\arg z) = \frac{Im\{z\}}{|z|}$$
	
	$Arg \, z$
	را برای مقدار مشخص و منحصر به فرد از
	$$-\pi < \arg z \leq \pi \quad or \quad 0\leq  \arg z < 2\pi$$
	به کار می‌بریم این مقدار 
	$\theta$
	به مقدار اصلی شناسه مرسوم است.
	
	برای هر عدد حقیقی 
	$\theta$
	داریم
	$$e^{i \theta} = \cos \theta + i \sin \theta$$
	برای عدد مختلط
	$z$
	 نیز برابر است یعنی
	 $e^{iz} = \cos z + i \sin z$
	 
	 نمایش قطبی اعداد مختلط:
	 $$z = x +iy = |z|(\cos (Arg z) + i sin(Arg z)) = r (\cos \theta + i \sin \theta) = |z|e^{iArg z} = re^{i\theta}$$
	 
	 \subsection{توان عدد مختلط}
	 برای عدد مختلط غیر صفر
	 $z = x + iy = r(\cos \theta + i \sin \theta)$
	 توان 
	 $n$
	 ام
	 به فرمول دموآر مدوف است به صورت زیر داریم
	 $$z^n = r^n(\cos n\theta + i \sin n\theta) = r^n \exp^{in\theta} \,,\, n \in \mathbb{Z}$$
	 
	 \textbf{مثال:}
	 مقدار 
	 $(1 - i)^{16}$
	 را بدست آورید.
	 $$1 - i = \sqrt{1^2 + (-1)^2} e^{i (-\frac{\pi}{4})} = \sqrt{2}e^{i (-\frac{\pi}{4})}$$
	 $$z^n = r^n(\cos n \theta + i\sin n\theta) \Rightarrow (1 - i)^{16} = \sqrt{2}^{16} (\cos (-\frac{-16 \pi}{4}) + i \sin (-\frac{-16 \pi}{4})) = 2^8(1+ 0i) = 2^8$$
	 \subsubsection{ریشه عدد مختلط}
	 عدد مختلط 
	 $z$
	 را درنظر بگیرید عدد مختلط 
	 $w$
	 را ریشه 
	 $n$
	 ام 
	 $z$
	 می‌گیریم هرگاه
	 $w^n = z\qquad$
	 $$z = re^{i \theta} = r(\cos \theta_0 + i sin \theta_0) $$
	 $$ w = \sqrt[n]{r}(\cos \frac{2k\pi + \theta_0}{n} + i \sin \frac{2k\pi + \theta_0}{n}) ; k = 0, 1, 2, \dots, n -1 $$
	 
	 \textbf{مثال:}
	 ریشه معادله 
	 $z^4 - 1 = i$
	 را بدست آورید.
	 $$z^4 = 1 + i = \sqrt{2}(\cos (\frac{ \pi}{4}) + i \sin (\frac{ \pi}{4}))$$
	 $$z = \sqrt[8]{2} (\cos \frac{2k\pi + \frac{ \pi}{4}}{4} + i \sin \frac{2k\pi + \frac{ \pi}{4}}{4}) ; k = 0, 1, 2, 3$$
	 $$z_0=\sqrt[8]{2} (\cos \frac{ \pi}{16} + i \sin \frac{ \pi}{16})$$
	 $$z_1 = (\cos \frac{ 2\pi + \frac{\pi}{4}}{4} + i \sin \frac{ 2\pi + \frac{\pi}{4}}{4}) = \sqrt[8]{2} (\cos \frac{\pi}{2}+\frac{ \pi}{16} + i \sin \frac{\pi}{2}+\frac{ \pi}{16})$$
	 $$ z_2 = z_3 $$
	 \section{همسایگی}
	 یک همسایگی عدد حقیقی
	 $x_0$
	 فاصله ای به شکل 
	 $(x_0 - r, x_0 + r)$
	 است که 
	 $r$
	 یک عدد حقیقی و مثبت است.
	 $$N_r(x_0 ) = \{x \in \mathbb{R}; |x-x_0| < r\} \subseteq \mathbb{R}$$
	 \section{نقطه درونی}
	 نقطه درونی 
	 $z_0$
	 را نقطه درونی 
	 $S \subseteq \mathbb{C}$
	 گوییم هر گاه همسایگی از 
	 $z_0$
	 داشته باشد که درون 
	 $S$
	 است.
	 \section{مجموعه باز}
	 مجموعه
	 $S \subseteq \mathbb{C}$
	 را باز گوییم هرگاه دو نقطه آن درونی باشد.
	 \subsection{نقطه  خارجی}
	 نقطه 
	 $z_0$
	 را نقطه خارجی 
	 	 $S \subseteq \mathbb{C}$
	 	 گوییم هرگاه یک همسایگی 
	 	 $z_0$
	 	 در مجموعه 
	 	 $S$
	 	 نباشد.
	 	 \subsection{نقطه مرزی}
	 	 نقطه 
	 	 $z_0$
	 	 را نقطه مرزی مجموعه 
	 	 	 	 $S \subseteq \mathbb{C}$
	 	 	 	 گوییم هرگاه نه نقطه داخلی و نه نقطه خارجی باشد.
	 	 	 	 \section{نقطه حدی}
	 	 	 	 نقطه 
	 	 	 	 $z_0$
	 	 	 	 را نقطه حدی 
	 	 	 	 	 	 $S \subseteq \mathbb{C}$
	 	 	 	 	 	 گوییم هرگاه
	 	 	 	 	 	 $$\forall r > 0 \quad N_r(z_0) \cap S \textbackslash \{z_0\} \neq \emptyset$$
	 	 	 	 	 	 \section{مجموعه بسته}
	 	 	 	 	 	 مجموعه
	 	 	 	 	 	 	 	 $S \subseteq \mathbb{C}$
	 	 	 	 	 	 	 	 بسته است هرگاه شامل همه نقاط حدی اش باشد.
	 	 	 	 	 	 	 	 \section{بستار مجموعه}
	 	 	 	 	 	 	 	 بستار مجموعه
	 	 	 	 	 	 	 	 	 	 	 	 	 	 $S \subseteq \mathbb{C}$
	 	 	 	 	 	 	 	 	 	 	 	 	 	 را با 
	 	 	 	 	 	 	 	 	 	 	 	 	 	 $\bar{S}$
	 	 	 	 	 	 	 	 	 	 	 	 	 	 نمایش می‌دهند و شامل نقاط
	 	 	 	 	 	 	 	 	 	 	 	 	 	 $S$
	 	 	 	 	 	 	 	 	 	 	 	 	 	 و نقاط حدی 
	 	 	 	 	 	 	 	$S$
	 	 	 	 	 	 	 	است.
	 	 	 	 	\section{صفحه مختلط توسعه یافته}
	 	 	 	 	$$\mathbb{C}^\star = \mathbb{C} \cup \{\mp \infty\}$$
	 	 	 	 	
	 	 	 	 	\textbf{مثال: }
	 	 	 	 	$\lim_{z \to z_0}f(z) = w_0$
	 	 	 	 	$$\forall \epsilon > 0 , \exists \delta > 0 , \forall z (|z - z_0|< \delta \Rightarrow |f(z) - w_0|<\epsilon)$$
	 	 	 	 	  	 $\lim_{z \to \infty} f(z) = w_0$	
	 	 	 	 	  	$$\forall \epsilon > 0 \quad \exists \delta>0 \quad \forall (\frac{1}{|z|} < \delta) \Rightarrow |f(z) - w_0| < \epsilon) $$ 
	 	 	 	 	  	$\lim_{z \to -\infty} = w_0$
	 	 	 	 	  		 	 	 	 در کامپلکس ها منفی نداریم چون ترتیب ندارد.
	 	 	 	 	 	 	 	 	 	 	 	 	 	 
	 	 	 	 
	 	 	 	 $$d(x, y) = |x| + |y| \qquad d(x, y) = |x+ y|$$
	 	 	 	 
	 	 	 	 $$\mathbb{C}^\star = \mathbb{C} \cup \{ \infty\} \qquad \lim_{z \to \infty} f(z) = w_0$$
	 	 	 	 $$\forall \epsilon > 0 , \exists \delta > 0 , \forall z (\frac{1}{|z|} < \delta \Rightarrow |f(z) - w_0| < \epsilon)$$
	 	 	 	 
	 	 	 	 
	 	 	 	 \textbf{مثال:}
	 	 	 	 $\lim_{z \to z_0} f(z) = \infty$ 
	 	 	 	 $$\forall > 0 \quad \exists \delta > 0 \quad \forall z , (|z-z_0|< \delta \Rightarrow \frac{1}{f(z)} < \epsilon)$$	 	 	 	 	 	 	 	 	 	 
	 	 	 	 \textbf{نکته:}
	 	 	 	 $$\lim_{z \to z_0} f(z) = \infty \Leftrightarrow \lim_{z \to z_0} \frac{1}{f(z)} = 0$$
	 	 	 	 $$\lim_{z \to \infty} f(z) = w_0 \Leftrightarrow \lim_{z \to 0} f(\frac{1}{z}) = w_0$$
	 	 	 	 $$\lim_{z \to \infty}f(z) = \infty \Leftrightarrow \lim_{z \to 0} f(\frac{1}{f(z)}= 0$$
	 	 	 	 
	 	 	 	 \section{نگاشت مختلط}
	 	 	 	 فرض کنید 
	 	 	 	 $S$
	 	 	 	 یک مجموعه باشد تابع مختلط 
	 	 	 	 $w$
	 	 	 	 را از متغیر های
	 	 	 	 $z = x = iy$
	 	 	 	 به صورت 
	 	 	 	 $f: S \subseteq \mathbb{C} \rightarrow \mathbb{C}$
	 	 	 	 $$w = f(z) = u(x, y) + iv(x, y)$$
	 	 	 	 $g:S \subseteq \mathbb{R} \rightarrow \mathbb{C}$
	 	 	 	 $$w = g(t) = x(t) + iy(t) = (x(t), y(t))$$
	 	 	 	 نمایش می‌دهیم.
	 	 	 	 
	 	 	 	 تابع 
	 	 	 	 $w = f(z)$
	 	 	 	 تک کقداری است اگر به ازای هر مقدار از 
	 	 	 	 $z$
	 	 	 	 در حوزه تعریف 
	 	 	 	 $S$
	 	 	 	 یک و تنها یک  مقدار  به 
	 	 	 	 $w$
	 	 	 	 نسبت داده می‌شود.
	 	 	 	 
	 	 	 	 مثال
	 	 	 	 \begin{enumerate}
	 	 	 	 	\item 
	 	 	 	 	تابع
	 	 	 	 	$w=f(z) = z^2$
	 	 	 	 	تک مقداریست.
	 	 	 	 	\item
	 	 	 	 	تابع 
	 	 	 	 	$Arg(z)$
	 	 	 	 	تک مقداری است.
	 	 	 	 	\item
	 	 	 	 	$Im\{z\} \quad, \quad Re\{z\} \quad , \quad |z|$
	 	 	 	 	تک مقداریست.
	 	 	 	 	
	 	 	 	 \end{enumerate}
 	 	 	 \textbf{تعریف:}
 	 	 	 تابع 
 	 	 	 $w = f(z)$
 	 	 	 چند مقداری است اگر برای بعضی یا تمام مقادیر 
 	 	 	 $z$
 	 	 	 در حوزه تعریف
 	 	 	 $S$
 	 	 	 ،
 	 	 	 مقادیر مختلفی به 
 	 	 	 $w$
 	 	 	 نسبت داده شود.
 	 	 	 
 	 	 	 مثال:
 	 	 	 \begin{enumerate}
 	 	 	 	\item 
 	 	 	 	تابع 
 	 	 	 	$w = f(z) = z^{\frac{1}{2}}\qquad f: \mathbb{C} \rightarrow \mathbb{C}$
 	 	 	 	$$z = i \qquad (i)^{\frac{1}{2}} = \mp \frac{\sqrt{2}}{2}(1 + i1)$$
 	 	 	 	\item
 	 	 	 	تابع چند مقداری
 	 	 	 	$\arg z  = 2k \pi + Arg z , \quad k= 0, \mp1, \mp 2 , \dots \qquad k \in \mathbb{Z}$
 	 	 	 	
 	 	 	 \end{enumerate}
  	 	 \chapter{فصل دوم}
  	 	 \section{حد}
 	 	 	 
	 	 	 	  فرض کنید تابع 
	 	 	 	  $w = f(z)$
	 	 	 	  در همه نقاط 
	 	 	 	  $z$
	 	 	 	  از یک همسایگی محذوف 
	 	 	 	  $z_0$
	 	 	 	  تعریف شده باشد حد تابع 
	 	 	 	  $w = f(z)$
	 	 	 	  در نقطه 
	 	 	 	  $z_0$
	 	 	 	  را با نماد 
	 	 	 	  $$\lim_{z \to z_0} f(z) = w_0$$
	 	 	 	  نمایش می دهیم و بدان معنی است که 
	 	 	 	  $$\forall \epsilon > 0 , \exists \delta > 0 , \forall z > 0 (|z - z_0| < \delta \Rightarrow |f(z) - w_0| < \epsilon)$$
	 	 	 	  
	 	 	 	  \textbf{نکته:}
	 	 	 	  وقتی که 
	 	 	 	  $z \to z_0$
	 	 	 	  ممکن است 
	 	 	 	  $z$
	 	 	 	  در امتداد مسیر های مختلف به 
	 	 	 	  $z_0$
	 	 	 	  نزدیک شود در صورت وجود حد
	 	 	 	  ،
	 	 	 	  حاصل تمامی حدود باهم برابر هستند.
	 	 	 	  \textbf{مثال:}
	 	 	 	  ثابت کنید نگاشت 
	 	 	 	  $f(z) = Arg z$
	 	 	 	  روی قسمتم منفی محور حقیقی حد ندارد.
	 	 	 	  
	 	 	 	  \textbf{اثبات:}
	 	 	 	  فرض کنید
	 	 	 	  $z_0 \in (- \infty , 0)$
	 	 	 	  
	 	 	 	  $$z_n = z_0 + \frac{i}{n} \quad ; \quad z' = z  - \frac{i}{n}$$
	 	 	 	  $$f(z_n) = Arg(z_n) = \arctan(\frac{1}{nz_0}) = n$$
	 	 	 	  $$f(z'_n) = Arg(Z_n') =\arctan(\frac{-1}{nz_0}) = -\arctan(\frac{1}{nz_0}) = -n$$	
	 	 	 	  
	 	 	 	  \section{پیوستگی}
	 	 	 	  تابع 
	 	 	 	  $f: S \subseteq \mathbb{C} \rightarrow \mathbb{C}$
	 	 	 	  در نقطه 
	 	 	 	  $z_0$
	 	 	 	  پیوسته است هرگاه
	 	 	 	  $$\lim_{z \to z_0} f(z) = f(z_0)$$
	 	 	 	  
	 	 	 	  \textbf{مثال}:
	 	 	 	  پیوستگی تابع زیر را در 
	 	 	 	  $z_0 = (0, 0)$
	 	 	 	  را بررسی کنید.
	 	 	 	  \[
	 	 	 	  f(z)=
	 	 	 	  \begin{cases}
	 	 	 	  	\frac{\bar{z}}{z} \qquad z \neq (0,0) \\
	 	 	 	  	1 \qquad z = (0,0)\\
	 	 	 	  \end{cases}
	 	 	 	  \]
	 	 	 	  
	 	 	 	  $$\lim_{z \to (0,0)} f(z) = \lim_{y \to 0} \frac{-iy}{iy} = -1$$
	 	 	 	  $$\lim_{z \to (0,0)} f(z) = \lim_{x \to 0} \frac{x}{x} = 1$$
	 	 	 	  حد نداریم.
	 	 	 	  
	 	 	 	  \begin{itemize}
	 	 	 	  	\item 
	 	 	 	  	روش دوم(قطبی):
	 	 	 	  	
	 	 	 	  	$$x = r\cos \theta , y = r\sin \theta$$
	 	 	 	  	$$\lim_{z \to (0,0)} f(z) = \lim_{r \to 0} \frac{r(\cos \theta - i\sin)}{r(\cos \theta + i \sin \theta)} = \lim_{r \to 0} \frac{(\cos \theta - i\sin)}{(\cos \theta + i \sin \theta)} = \frac{(\cos \theta - i\sin)}{(\cos \theta + i \sin \theta)}$$
	 	 	 	  	حد ندارد.
	 	 	 	  	\section{پیوستگی یکنواخت}
	 	 	 	  	تابع
	 	 	 	  	$w = f(z)$
	 	 	 	  	را در ناحیه 
	 	 	 	  	$\mathbb{R}$
	 	 	 	  	پیوسته یکنواخت می‌گوییم
	 	 	 	  	$$\forall \epsilon > 0 \quad \exists \delta > 0  \quad \forall z_1, z_2 (|z_1 - z_2| < \delta \Rightarrow |f(z_1) - f(z_2)| < \epsilon)$$
	 	 	 	  	
	 	 	 	  	\textbf{مثال}:
	 	 	 	  	تابع 
	 	 	 	  	$f(z) = \frac{1}{z}$
	 	 	 	  	روی مجموعه 
	 	 	 	  	$R = \{z \in \mathbb{C} ; 0<|z|<1\}$
 	 	 	  	پیوسته یکنواخت نیست.
 	 	 	  	
 	 	 	  	\textbf{مثال:}
 	 	 	  	تابع
	 	 	 	  		 	 	 	  	$f(z) = \frac{1}{z}$
	 	 	 	  	روی مجموعه
	 	 	 	  	$$R_\eta = \{z \in \mathbb{C} ; |z|\geq \eta , \eta > 0\}$$
	 	 	 	  	پیوسته یکنواخت هست.
	 	 	 	  \end{itemize}
 	 	 	  
 	 	 	  \section{مشتق}
 	 	 	  فرض کنید تابع
 	 	 	  $f(z) = w$
 	 	 	  را در دامنه 
 	 	 	  $D \subseteq \mathbb{C}$
 	 	 	  دراین صورت مشتق
 	 	 	  $f(z)$
 	 	 	  در نقطه
 	 	 	  $z_0$
 	 	 	  که آن را با 
 	 	 	  $f'(z_0)$
 	 	 	  نمایش می‌دهیم به صورت زیر تعریف می‌شود.
 	 	 	  $$f'(z_0) =\lim_{z \to z_0} \frac{f(z) - f(z_0)}{z-z_0}$$
 	 	 	  
 	 	 	  \textbf{مثال:}
 	 	 	  نشان دهید تابع
 	 	 	  $f(z) = \bar{z}$
 	 	 	  در صفحه مختلط مشتق پذیر نیست.
 	 	 	  
 	 	 	  فرض کنید
 	 	 	  $z_0 \in \mathbb{C}$
 	 	 	  دلخواه باشد 
 	 	 	  $z = x+iy \quad , \quad z_0 = x_0 + iy_0\quad$
 	 	 	  
 	 	 	  $$f'(z_0) = \lim_{z \to z_0} \frac{f(z) - f(z_0)}{z-z_0} = \lim_{z \to z_0} \frac{\bar{z} - \bar{z_0}}{z - z_0}$$
 	 	 	  
 	 	 	  $$\overset{x = x_0 , z = x_0 + iy}{=} \lim_{y \to y_0} \frac{x_0 - iy - x_0 + iy_0}{x_0 - iy - x_0 - iy_0} = \lim_{z \to z_0} \frac{-iy+iy_0}{iy - iy_0} = -1$$
 	 	 	  
 	 	 	  $$\lim_{z \to z_0} \frac{\bar{z} - \bar{z_0}}{z - z_0}\overset{y = y_0 , z = x + iy_0}{=} \lim_{x \to x_0} \frac{x- iy_0 - x_0 + iy_0}{x - iy_0 - x_0 - iy_0} = \lim_{x \to x_0}\frac{x - x_0}{x - x_0} = 1$$
 	 	 	  
 	 	 	  \section{معادلات کشی-ریمان}
 	 	 	  تابع
 	 	 	  $w = f(z) = u(x, y) + i v(x, y)$
 	 	 	  در نقطه
 	 	 	  $z_0 = x_0 + iy_0$
 	 	 	  مشتق پذیر باشد آنگاه معادلات زیر برقرار است.
 	 	 	  \[
 	 	 	  \begin{cases}
 	 	 	  	\frac{\partial u}{\partial x} (z_0) = \frac{\partial v}{\partial y} (z_0)\\
				\frac{\partial u}{\partial y} (z_0) = \frac{\partial v}{\partial x} (z_0)
 	 	 	  \end{cases}
 	 	 	  \]
 	 	 	  شرط لازم مشتق پذیری در نقطه 
 	 	 	  $z_0=(x_0, y_0)$
 	 	 	  \[
 	 	 	  \begin{cases}
 	 	 	  	u_x(z_0) = v_y(z_0)\\
 	 	 	  	u_y(z_0) = v_x(z_0)
 	 	 	  \end{cases}
 	 	 	  \]
 	 	 	  مشتق تابع در نقطه 
 	 	 	  $z_0$
 	 	 	  به صورت 
 	 	 	  $$f'(z_0) = \frac{\partial u}{\partial x}(z_0) + i\frac{\partial v}{\partial x}(z_0) = \frac{\partial f}{\partial x}(z_0)$$
 	 	 	  
 	 	 	  یا
 	 	 	  $$f'(z_0) = \frac{\partial v}{\partial y}(z_0) - i\frac{\partial u}{\partial y}(z_0) = \frac{1}{i}\frac{\partial f}{\partial y}(z_0)$$
 	 	 	  
 	 	 	  \textbf{مثال:}
 	 	 	  برای تابع
 	 	 	  $$f(z) = \bar{z} = x - iy \quad, \quad u(x, y) = x , v(x, y) = y$$
 	 	 	  معادلات  کشی -ریمان در نقطه دلخواه 
 	 	 	  $z=z_0$
 	 	 	  به صورت زیر است:
 	 	 	  $$u(x, y) = x \quad,\quad v(x, y) = -y$$
 	 	 	  $$\frac{\partial u}{\partial x}=1 \quad, \quad\frac{\partial v}{\partial y} = -1$$
 	 	 	  $$\frac{\partial u}{\partial y} = 0\quad , \quad \frac{\partial v}{\partial x} = 0$$
 	 	 	  شرط لازم مشتق پذیر را ندارد پس در هیچ نقطه ای از 
 	 	 	  $\mathbb{C}$
 	 	 	  مشتق پذیر نیست.
 	 	 	  \textbf{مثال:}
 	 	 	  \[
 	 	 	  f(z) = 
 	 	 	  \begin{cases}
 	 	 	  	\frac{(1+i)xy}{x^3+y^3} \qquad z \neq 0\\
 	 	 	  	0 \qquad z = 0
 	 	 	  \end{cases}
 	 	 	  \]
 	 	 	  مشتق پذیری تابع فوق را در 
 	 	 	  $z=0$
 	 	 	  بررسی کنید.
 	 	 	  
 	 	 	  \[
 	 	 	  f(z) = 
 	 	 	  \begin{cases}
 	 	 	  	\frac{xy}{x^3+y^3} + i \frac{xy}{x^3+y^3} \qquad z \neq 0\\
 	 	 	  	0 \qquad z = 0
 	 	 	  \end{cases}
 	 	 	  \]
 	 	 	  $$u(x, y) = \frac{xy}{x^3+y^3} \quad , \quad v(x, y) = \frac{xy}{x^3+y^3}$$
 	 	 	  $$\frac{\partial u}{\partial x}(0, 0) = \lim_{x \to 0} \frac{u(x, 0) - u(0,0)}{x} = \lim_{x \to 0} \frac{0 - 0}{x} = 0$$
 	 	 	  $$\frac{\partial v}{\partial y}(0, 0) = \lim_{y \to 0} \frac{v(0, y) - v(0,0)}{y} = \lim_{y \to 0} \frac{0 - 0}{y} = 0$$
 	 	 	  $$\frac{\partial u}{\partial y}(0, 0) = \lim_{y \to 0} \frac{u(0, y) - u(0,0)}{y} = \lim_{y \to 0} \frac{0 - 0}{y} = 0$$
 	 	 	  $$\frac{\partial v}{\partial x}(0, 0) = \lim_{x \to 0} \frac{v(x, 0) - v(0,0)}{x} = \lim_{x \to 0} \frac{0 - 0}{x} = 0$$
 	 	 	  
 	 	 	  $$f'(0,0) = \lim_{z \to (0,0)} \frac{f(z) - f(0, 0)}{z} = \lim_{(x, y) \to (0, 0)} \frac{\frac{(1+i)xy}{x^3+y^3} - 0}{x+iy} = \lim_{(x, y) \to (0, 0)} \frac{\frac{(1+i)xy}{x^3+y^3} - 0}{x+iy} \overset{x=y}{=}$$
 	 	 	  $$\lim_{z \to (0,0)} \frac{(1+i)xy}{(x^3+y^3)(x+iy)} = \lim_{x \to 0} \frac{(1 + i) x^2}{2x^3(1+i)x} = \lim_{x \to 0} \frac{(1+i)}{2x^2(1+i)} = \infty$$
 	 	 	  تابع در 
 	 	 	  $(0,0)$
 	 	 	  مشتق پذیر نیست.
 	 	 	  
 	 	 	  \section{معادلات کشی-ریمان در مختصات قطبی}
 	 	 	  از آنجایی که
 	 	 	  $x = r\cos \theta$
 	 	 	  و
 	 	 	  $y = r\sin \theta$
 	 	 	  می‌باشد اگر تابع 
 	 	 	  $w = f(z) =u(x, y) + iv(x, y)$
 	 	 	  در نقطه 
 	 	 	  $z_0 = x_0 + iy_0$
 	 	 	  مشتق پذیر باشد آنگاه
 	 	 	  \[
 	 	 	  \begin{cases}
 	 	 	  	\frac{\partial u}{\partial r} = \frac{1}{r} \frac{\partial v}{\partial \theta} (z_0)\\
 	 	 	  	\frac{\partial v}{\partial r} = -\frac{1}{r} \frac{\partial u}{\partial \theta} (z_0)
 	 	 	  	
 	 	 	  \end{cases}
 	 	 	  \] 
 	 	 	  و مشتق تابع در نقطه 
 	 	 	  $z_0$
 	 	 	  به صورت 
 	 	 	  $$f'(z_0) = e^{i \theta}(\frac{\partial u}{\partial r}(z_0) + i\frac{\partial v}{\partial r}(z_0)) = e^{-i\theta}(\frac{\partial f}{r}(z_0))$$
 	 	 	  
 	 	 	  
 	 	 	  \textbf{مثال:}
 	 	 	  برای تابع
 	 	 	  $f(z) = \ln z$
 	 	 	  با
 	 	 	  $-\pi < Arg z < \pi$
 	 	 	  و 
 	 	 	  $r>0$
 	 	 	  ،
 	 	 	  معادلات کشی-ریمان در نقطه دلخواه
 	 	 	  $z = re^{i\theta} \in \mathbb{C} \% (-\infty ,  0)$
 	 	 	  بررسی کنید.
 	 	 	  $$f(z) = \ln z = \ln (re^{i\theta}) = \ln r + \ln (e^{i\theta}) = \ln r + i\theta$$
 	 	 	  که
 	 	 	  $$u(r, \theta) = \ln r \quad , \quad v(r, \theta) = i\theta$$
 	 	 	  
 	 	 	  $$\frac{\partial u}{\partial r} = \frac{1}{r} \quad, \quad \frac{\partial v}{\partial \theta} = 1$$
 	 	 	  $$\frac{\partial u}{\partial r} = \frac{1}{r} = \frac{1}{r} \frac{\partial v}{\partial \theta} $$
 	 	 	  $$\frac{\partial v}{\partial r} = 0 \quad , \quad \frac{\partial u}{\partial \theta} = = 0$$
 	 	 	  $$f'(z_0) = e^{-i\theta}(\frac{\partial f}{\partial r]}(z_0)) = e^{-i \theta_0}(\frac{1}{r_0} + ix_0) = e^{-i \theta_0} \frac{1}{r_0} = \frac{1}{r_0e^{i\theta_0}} = \frac{1}{z_0}$$
 	 	 	  
 	 	 	  \section{تابع تحلیلی}
 	 	 	  تابع 
 	 	 	  $w= f(z)$
 	 	 	  در نقطه 
 	 	 	  $z$
 	 	 	  تحلیلی گوییم هرگاه یک همسایگی از این نقطه 
 	 	 	  
 	 	 	  $z$
 	 	 	  وجود داشته باشد به طوریکه در همه جا این همسایگی مشتق پذیر باشد.
 	 	 	  
 	 	 	  \textbf{تعریف تابع تام:}
 	 	 	  تابعی که در همه جای دامنه تابع تحلیلی باشد تابع تام نامیده می‌شود.
 	 	 	  
 	 	 	  \textbf{قضیه(شرط کافی تحلیلی بودن):}
 	 	 	  فرض کنید
 	 	 	  $f(z) = u(x, y) + iv(x, y)$
 	 	 	  در دامنه 
 	 	 	  $D$
 	 	 	  تعریف شده و توابع 
 	 	 	  $ u(x, y) , 	v(x, y)$
 	 	 	  دارای مشتقات جزئی پیوسته باشند و در تمام نقاط دامنه 
 	 	 	  $f$
 	 	 	  ،
 	 	 	  معادلات کشی-ریمان برقرار باشد در این صورت 
 	 	 	  $f(z)$
 	 	 	  در $D$
 	 	 	  تحلیلی است.
 	 	 	  
 	 	 	  \section{نقطه تکین}
 	 	 	  نقطه 
 	 	 	  $z_0$
 	 	 	  را تکین تابع 
 	 	 	  $w = f(z)$
 	 	 	  گوییم هرگاه در 
 	 	 	  $z_0$
 	 	 	  تحلیلی باشد و در نقطه ای از همسایگی
 	 	 	  $z_0$
 	 	 	  تحیلی باشد.
 	 	 	  \textbf{مثال:}
 	 	 	  $z= 0 $
 	 	 	  برای تابع
 	 	 	  $\frac{1}{z}$
 	 	 	  یک نقطه تکین است.	 	 
 	 	 	  
 	 	 	  \section{تابع همساز}
 	 	 	   هر تابع حقیقی
 	 	 	   \[
 	 	 	   \begin{cases}
 	 	 	   	u : \mathbb{R}^2 \rightarrow \mathbb{R}\\
 	 	 	   	(x, y) \rightarrow u(x, y)
 	 	 	   \end{cases}
 	 	 	   \]
 	 	 	   که دارای مشتقات جزئی مرتبه اول و دوم پیوسته باشد و در معادلات لاپلاس
 	 	 	   $\frac{\partial^2u}{\partial x^2}+ \frac{\partial^2u}{\partial y^2} = 0$
 	 	 	   صدق کند تابع همساز نامیده می‌شود.
 	 	 	   
 	 	 	   \textbf{قضیه:}
 	 	 	   اگر تابع 
 	 	 	   $f(z) = u(x, y) + iv(x, y)$
 	 	 	   در دامنه 
 	 	 	  $D$
 	 	 	  تحلیلی باشد توابع  مولفه ای آن یعنی 
 	 	 	  $u, v$
 	 	 	  در 
 	 	 	  $D$
 	 	 	  همساز هستند.
 	 	 	  
 	 	 	  \textbf{تعریف:}
 	 	 	  
 	 	 	  $v$
 	 	 	  مزدوج همسازی از 
 	 	 	  $u$
 	 	 	  نامیده می‌شود.
 	 	 	  
 	 	 	  
 	 	 	  \section{رابطه همسازی برای تابع تحلیلی در نمایش قطبی}
 	 	 	  اگر تابع
 	 	 	  $f(z) = u(r, \theta) + iv(r, \theta)$
 	 	 	  در دامنه 
 	 	 	  $D$
 	 	 	  تحلیلی باشد آنگاه
 	 	 	  $$ \frac{\partial^2 u}{\partial r^2} + \frac{1}{r}\frac{\partial u}{\partial r}+\frac{1}{r^2}\frac{\partial^2 u}{\partial \theta^2} = 0$$
 	 	 	  
 	 	 	  \textbf{مثال:}
 	 	 	  نشان دهید
 	 	 	  $u = 3x^2y + 2x^2-y^3 - 2y^2$
 	 	 	  یک تابع همساز است. تابع مزدوج همسازی آن را بدست آورید.
 	 	 	  $$\frac{\partial u}{\partial x} = 6xy + 4x \Rightarrow \frac{\partial^2 u}{\partial x^2} = 6y+4$$
 	 	 	  $$\frac{\partial u}{\partial y} = 3x^2-3y^2 - 4y \Rightarrow \frac{\partial^2u}{\partial y^2} = -6y-4$$
 	 	 	  در معادلات لاپلاس صدق می‌کند.
 	 	 	  
 	 	 	  فرض کنید
 	 	 	  $v$
 	 	 	  مزدوج همساز 
 	 	 	  $u$
 	 	 	  است در اینصورت
 	 	 	  $f(z) = u(x, y) + iv(x, y)$
 	 	 	  یک تابع تحلیلی است پس در معادلات کشی-ریمان صدق می‌کند
 	 	 	  $$\frac{\partial u}{\partial x} = \frac{\partial v}{\partial y} = 6xy+4x\Rightarrow v(x, y) = \int (6xy + 4x) dy = 3xy^2 + 4xy + g(x)$$
 	 	 	  $$\frac{\partial u}{\partial x} = - \frac{\partial v}{\partial x} = 3x^2 - 3y^2 - 4y$$
 	 	 	  
 	 	 	  $$\frac{\partial v}{\partial x} = -3x^2 + 3y^2 + 4y = 3y^2 + 4y + g'(x) \Rightarrow g'(x) = -3x^2 \Rightarrow g(x) = \int -3x^2 \, dx$$
 	 	 	  
 	 	 	  $$g(x) = -x^3 + c$$
 	 	 	  $$v(x, y) = 3xy^2 + 4xy - x^3 + c$$
 	 	 	  \section{توابع مقدماتی}
 	 	 	  \subsection{تابع نمایی}
 	 	 	  تابع نمایی را به ازای هر عدد مختلط
 	 	 	  $z = x+iy$
 	 	 	  به فرم
 	 	 	  $\exp(z) = e^z = e^x(\cos y + i\sin y) = e^x \cos y + e^x \sin y$
 	 	 	  تعریف می‌کنیم. تابع نمایی 
 	 	 	  $e^z$
 	 	 	  در شرط کشی-ریمان صدق می‌کند و مشتقات جزئی پیوسته دارد بنابراین در هر نقطه از صفحه مختلط مشتق پذیر است
 	 	 	  $(e^z)' = e^z$
 	 	 	  لذا تابع تحلیلی است.
 	 	 	  
 	 	 	  \textbf{مثال:}
 	 	 	  $\exp(\bar z) = e^{\bar z} = e^x(\cos y - i\sin y) = e^x \cos y - i \sin y$
 	 	 	  
 	 	 	  \[
 	 	 	  \begin{cases}
 	 	 	  	\frac{\partial u}{\partial x} = e^x \cos y\\
 	 	 	  	\frac{\partial v}{\partial y} = -e^x \cos y
 	 	 	  \end{cases}
 	 	 	  \]
 	 	 	  پس شرط لازم مشتق پذیر ندارد پس تحلیلی نیست.
 	 	 	  \subsection{خواص تابع نمایی}
 	 	 	  \begin{enumerate}
 	 	 	  	\item 
 	 	 	  	برای هر عدد مختلط
 	 	 	  	$z = x + iy$
 	 	 	  	داریم: 
 	 	 	  	$e^z \neq 0$
 	 	 	  	\item
 	 	 	  	$|e^z| = e$
 	 	 	  	
 	 	 	  	$$\arg(e^z) = y + 2k\pi , k \in \mathbb{Z}$$
 	 	 	  	\item
 	 	 	  	$$e^{2\pi i } = 1 \quad, \quad e^{z +2\pi i} = e^z$$
 	 	 	  	یعنی تابع
 	 	 	  	$e^z$
 	 	 	  	متتناوب است و دوره متناوب آن 
 	 	 	  	$2 \pi i$
 	 	 	  	
 	 	 	  	$$\arg(e^z) = \tan^-1(\frac{Im(e^z)}{Re(e^z)}) = \tan^-1(\frac{e^x \sin y}{e^x \cos y }) = y$$
 	 	 	  	
 	 	 	  \end{enumerate}
 	 	 	  
 	 	 	  \subsection{توابع مثلثاتی}
 	 	 	  برای هر عدد مختلط
 	 	 	  $z$
 	 	 	  فرمول اویلر به صورت زیر است
 	 	 	  $$e^{iz} = \cos(z)+i \sin(z)$$
 	 	 	  در این صورت  برای هر 
 	 	 	  $z$
 	 	 	  در صفحه مختلط با توجه به فرمول اویلر داریم:
 	 	 	  $$e^{\mp iz} = \cos z \mp i\sin z$$
 	 	 	  $$\cos z = \frac{e^{iz} + e^{-iz}}{2} \qquad \sin z = \frac{e^{iz} - e^{-iz}}{2i}$$
 	 	 	  
 	 	 	  \subsection{خواص توابع مثلثاتی}
 	 	 	  \begin{enumerate}
 	 	 	  	\item 
 	 	 	  	توابع 
 	 	 	  	$\sin z$
 	 	 	  	و
 	 	 	  	$\cos z$
 	 	 	  	در صفحه مختلط تحلیلی است.
 	 	 	  (چون تابع 
 	 	 	  $e^z$
 	 	 	  تحلیلی است)
 	 	 	  \item
 	 	 	  $(\sin z)' = \cos z \qquad (\cos z)' = - \sin z$
 	 	 	  \item
 	 	 	  $$\sin^2 z + \cos^2 z = 1$$
 	 	 	  \item
 	 	 	  $$\sin (i\theta) = i \sinh \theta$$
 	 	 	  $$\sin (i\theta) = \frac{e^{i(i\theta)} - e^{-i(i\theta)}}{2i} = \frac{e^{-\theta} - e^{+\theta}}{2i} = i\sinh \theta$$
 	 	 	  \item
 	 	 	  $$\cos (i\theta) =  \cosh \theta$$
 	 	 	  \item
 	 	 	  $$\sin(z) = \sin(x + iy)  = \sin x \cos(iy)  +cos x \sin(iy) = \sin x \cosh y + i \cos x \sinh y$$
 	 	 	  $$\qquad Re(\sin z) = \sin x \cosh y \qquad Im(\sin z) = i\cos x \sinh y$$
 	 	 	  
 	 	 	  
 	 	 	  \end{enumerate}
  	 	  \subsection{توابع هایپربولیک}
  	 	  $$\sinh z = \frac{e^z - e^{-z}}{2} \qquad \cosh z = \frac{e^z + e^{-z}}{2}$$
  	 	  \subsection{خواص توابع هایپربولیک}
  	 	  \begin{enumerate}
  	 	  	\item 
  	 	  	$\sinh z$
  	 	  	و
  	 	  	$\cosh z$
  	 	  	در صفحه مختلط تحلیلی است.
  	 	  	\item
  	 	  	$$(\cosh z)' = \sinh z \qquad (\sinh z)' = \cosh z$$
  	 	  	\item
  	 	  	$$\cosh^2 - \sinh^ z = 1$$
  	 	  	\item
  	 	  	$$\cosh(iz) = \cos z$$
  	 	  	\item
  	 	  	$$\sinh(z) = \sinh(x + iy) =\sinh x \cos y+i \cosh x \sin y$$
  	 	  	\item
  	 	  	$$\cosh(z) = \cosh(x + iy) =\cosh x \cos y+i \sinh x \sin y$$
  	 	  \end{enumerate}
		\subsection{تابع لگاریتم}
		لگاریتم را می‌توان به عنوان مقداری چون 
	$w$
	به طوریکه 	
	$e^w = z$
	باشد، تعریف کرد.
	
	اما این تعریف با توجه به متناوب بودن تابع نمایی، وجود لگاریتم منحصر به فرد را خنثی می‌کند. زیرا اگر
	$e^w = z$
	آنگاه برای هر عدد صحیح 
	$k$
	داریم:
	$$e^{w + 2k \pi i} = z$$
	از این رو لگاریتم عدد مختلط 
	$z$
	را که با 
	$\ln z$
	 نمایش می‌دهیم، به صورت مجموعه ای از همه مقادیر 
	 $$w = \ln z \quad , \quad e^w = z$$
	 باشد بنابراین 
	 $$Ln (z) = \{w \in \mathbb{C} ; e^w = z\}$$
	 $L$
	 بزرگه چون چند مقداری است.
	 
	 فرض کنیم 
	 $z = re^{i(\theta + 2k \pi)}$
	 که 
	 $k$
	 عدد صحیح و 
	 $r \neq 0$
	 است. از معادله 
	 $e^w = z$
	 که 
	 $w = u + iv$
	 داریم 
	 $$e^w = z \rightarrow e^{u + iv} = re^{i(\theta + 2k \pi)} \qquad e^u = r , v = \theta + 2k \pi ;k \in \mathbb{Z}$$
	 بنابراین
	 $w = u + iv = Ln (z) = \ln r + i(\theta + 2k\pi)\quad ;\quad k \in \mathbb{Z}$
	 
	 $$w = Ln (z) = \ln |z| + i \arg z$$
	 از آنجایی که 
	 $arg z$
	 چند مقداری است تابع
	 $Ln(z)$
	 چند مقداری می‌باشد. با انتخاب 
	 
	 $\arg z = Arg z$
	 لگاریتم
	 $$Ln z = \ln z = \ln |z| + i Arg z$$
	 که در آن 
	 $-\pi < Arg z \leq \pi$
	را لگاریتم اصلی می‌نامیم.
	
	\textbf{مثال:}
	$$Ln 1 = \{w \in \mathbb{C} ; e^w = 1\} = \{2k\pi i;k \in \mathbb{Z}\}$$
	$$\ln 1 = \{2k\pi i; k = 0\} = 0$$
	
	\textbf{مثال:}
	$$Ln(-1) = \ln |-1| + i arg(-1) = \pi i + 2k\pi i ; k \in \mathbb{Z}$$
	$$\ln (-1) = \pi i $$
	
	\textbf{مثال:}
	$$Ln(1-\sqrt{-3}) = \{\ln 2 + i (2k\pi - \frac{\pi}{3}); k \in \mathbb{Z}\}$$
	$$\ln(1-\sqrt{-3}) = \ln 2 - \frac{\pi}{3}$$
	
	\textbf{نکته:}
	از آنجاییکه تابع تک مقداری 
	$Arg z$
	در
	$$\mathbb{C} \backslash (-\infty, 0] = \{z \in \mathbb{C}; z + |z| \neq 0\} $$
	پیوسته است لذا تابع لگاریتم اصلی 
	$$\ln z = \ln |z| + iArg z \qquad -\pi < Arg z < \pi$$
	در  
	$\mathbb{C} \backslash (-\infty, 0]$
	پیوسته است.
	
	مشتق تابع لگاریتم اصلی برای هر 
	$z \in \mathbb{C} \backslash (-\infty, 0 ]$
	تابع 
	$$\ln z = \ln |z| + i Arg z = \ln r + i \theta = u(r, \theta) + iv(r, \theta) \qquad \ln r  = u(r, \theta) , \theta =  v(r, \theta)$$
	\[
	\begin{cases}
		u_r = \frac{1}{r} = v_\theta\\
		\theta_r = -\frac{1}{r} u_\theta
	\end{cases}
	\]
	\[
	\left.
	\begin{array}{l}
		u_r = \frac{1}{r} \\
		u_\theta = 1
	\end{array}
	\right\} \Rightarrow \frac{1}{r} = \frac{1}{r}
	\]
	$$u_r = 0 , u_\theta = 0 \Rightarrow v_r = -\frac{1}{r}$$
	
	\[
	\frac{d}{dz} \ln z = e^{-i\theta} \frac{\partial f}{\partial r} = e^{-i\theta(u_r + iv_r)} = e^{-i\theta(\frac{1}{r} + 0)} = \frac{1}{r} e^{-i\theta} = \frac{1}{re^{i\theta}} = \frac{1}{z}
	\]
	
	\subsection{نمایی کسری}
	برای دو عدد صحیح و مثبت
	$m$
	و
	$n$
	که نسبت به هم اول باشند، تعریف می‌کنیم.
	$$(z^{\frac{1}{n}})^m = e^{\frac{m}{n} Ln z} = e^{\frac{m}{n}}(\ln |z| + i(Arg z + 2k \pi )) = e \qquad z \neq 0 , k = 0, 1, 2, \dots, n-1$$
	دارای 
	$n$
	مقدار متمایز می‌باشد.
	
	فرض کنید
	$c$
	یک عدد مختلط باشد تعریف می‌کنیم
	$$z^c = e^{c Lnz} = e^{c(\ln |z| + i (Arg z + 2k\pi))} = |z|^2 e^{i(Argz)} e^{i(2k\pi)} \quad z\neq 0 , k \in \mathbb{Z} $$
	
	 \textbf{نکته:}
	 اگر 
	 $C$
	 گویا نباشد، آنگاه 
	 $z^c$
	 دارای بی‌نهایت مقدار است.
	 \textbf{مثال:}
	 مقادیر
	 $5^\frac{1}{2}$
	 را بدست آورید.
	 $$	 5^\frac{1}{2} = e^{\frac{1}{2} Ln 5} = e^{\frac{1}{2}(\ln 5 + i(Arg 5 + 2k\pi))} = \sqrt{5} e^{k\pi i } = \mp \sqrt{5} \quad;\quad k = 0, 1$$
	 
	 \textbf{مثال:}
	 مقادیر
	 $i^i$
	 را بدست آورید.
	 $$ e^{i Ln i} = e^{i(\ln |i| + i(Arg 5 + 2k\pi))} =  e^{\frac{\pi}{2}+2k\pi i } \quad;\quad k \in \mathbb{Z}$$
	 
	 تابع
	 $z^\frac{1}{n}$
	 یک تابع 
	 $n$
	 مقداری در ریشه 
	 $n$
	 ام 
	 $z$
	 است یعنی
	 $$z^\frac{1}{n} = e^{\frac{1}{n} Ln z} = e^{\frac{1}{n}(\ln |z| + i(Arg z + 2k\pi))} = |z|^\frac{1}{n} e^{i\frac{Arg(z)}{n}} e^{i\frac{2k\pi}{n}} \quad;\quad k = 0, 1, 2, \dots, n-1$$
	 
	 \section{تبدیل خطی}
	 $$w = az + b$$
	 \begin{enumerate}
	 	\item 
	 	اگر 
	 	$a=1$
	 	یک انتقال داریم و شکل در صفحه 
	 	$w$
	 	همانند شکل در صفحه 
	 	$z$
	 	است اما نسبت به مرکز مختصات به صورت متفاوت  جایگذاری شده اند.
	 	\item
	 	اگر 
	 	$z_1 \rightarrow w_1 \qquad z_2 \rightarrow w_2$
	 	آنگاه
	 	$$|z_1 - z_2| = |w_1 - w_2| \qquad \arg(z_1- z_2) = \arg(w_1 - w_2)$$
	 	\item
	 	اگر
	 	$a\neq 0$
	 	و
	 	$b =0 $
	 	آنگاه
	 	\begin{enumerate}
	 		\item 
	 		اگر 
	 		$a$
	 		حقیقی باشد آنگاه 
	 		$$|w| = |a||z| \quad, \quad \arg w = \arg z$$
	 		اگر 
	 		$a>1$
	 		یک انبساز داریم
	 		
	 		اگر 
	 		$a<1$
	 		یک انقباض داریم.
	 		\item
	 		اگر
	 		$a$
	 		مختلط باشد یعنی
	 		$$a = |a| e^{i\alpha}$$
	 		آنگاه نگاشت شامل یک دوران به اندازه
	 		$\alpha$
	 		حول مبدا یک اقباض یا انبساط است.
	 		
	 		نگاشت 
	 		
	 	\end{enumerate}
	 \end{enumerate}
	 $w = \frac{1}{z}$
	 این نگاشت تناظری یک به یک بین نقاط غیر صفر صفحه 
	 $z$
	 و نقاط صفحه 
	 $w$
	 برقرار می‌کند با فرض
	 $z = re^{i\theta}$
	 $$w = \frac{1}{z}= \frac{1}{r} e^{-i\theta}$$
	 
	 \textbf{مثال:}
	 نگاشت
	 $\frac{1}{z}$
	 هر دایره با خط راست را به یک دایره با خط راست تضویر می‌کند.
	 \begin{equation}\label{eq1}
	 	A(x^2+y^2) + Bx + Cy + D = 0
	 \end{equation}
	 
	 $$w = u(x, y) + iv(x, y)$$
	 با توجه به نگاشت 
	 $w = \frac{1}{z}$
	 داریم:
	 $$z = \frac{1}{w} = \frac{1}{u + iv}=\frac{1 \times (u - iv)}{(u+iv)(u - iv)} = \frac{u - iv}{u^2+v^2} = x +iy$$
	 $$x = \frac{u}{u^2 + v^2}\,,\, y = \frac{-v}{u^2 + v^2}$$
	 در این صورت ب جایگذاری تساوی های فوق در 
	 \ref{eq1}
	 خواهیم داشت.
	 $$D(u^2 + v^2) + Bu -Cu + A = 0$$
	 که معادله کلی خط یا دایره در صفحه 
	 $w$
	 است.
	 \chapter{انتگرال}
	 \section{خم پیوسته یا کمان}
	 شکل پارامتری یک خم پیوسته یا کمان به صورت 
	 \[
	 \begin{cases}
	 	x = \varphi(t) \\
	 	y = \psi(t) 
	 \end{cases}
 \qquad a \leq t \leq b
	 \]
	 می‌باشد که 
	 $\varphi , \psi$
	 در فاصله 
	 $[a, b]$
	 پیوسته هستند و با فرض پیوستگی 
	 $\varphi$
	 و
	 $\psi$
	 کمان
	 $$z(t) = \varphi(t) + i\psi(t); \quad a \leq t \leq b$$
	 منحنی پیوسته ای را تعریف می‌کند که در صفحه 
	 $z$
	 نقطه 
	 $A = z(a)$
	 را به نقطه
	 $B = z(b)$
	 وصل می‌کند.
	 \textbf{مثال:}
	 خط شکسته‌
	 \[
	 z(t) = 
	 \begin{cases}
	 	t + it \qquad 0\leq t\leq 1 \\
	 	t + i \qquad 1\leq t \leq 2
	 \end{cases}
	 \]
	 متشکل از پاره خطی که از 
	 $0$
	 تا 
	 $1 + i$
	 و به دنبال آن پاره‌خطی از 
	 $1+i$
	 تا
	 $2 + i$
	 یک خم پیوسته یا کمان است.
	 \section{منحنی ساده}
	 اگر منحنی پیوسته یا کمان
	 $$z(t) = \varphi(t)+ i\psi(t);\quad a \leq t \leq b$$
	 خودش را قطع نکند و یا خودش مماس نباشد.
	 
	 اگر 
	 $t_1 \neq t_2$
	 داشته باشیم
	 $$z(t_1) \neq z(t_2)$$
	 آن را منحنی ساده یا کمان جردن می‌نامیم.
	 
	 \textbf{مثال:}
	 $z(t) = t + i\ln(1 + t) \qquad 0\leq t \leq 1$
	 یک منحنی ساده یا کمان جردن است که 
	 $A = z(0) = 1$
	 را به نقطه
	 $B = z(1) + 1 + i\ln 2 $
	 وصل می‌کند.
	 \section{کمان ساده هموار}
	 خم پیوسته یا کمان 
	 $z(t) = \varphi(t) + i\psi(t); \quad a \leq t \leq b$
	 را کمان ساده هموار گوییم، هرگاه توابع 
	 $\varphi, \psi$
	 دارای مشتقات پیوسته در 
	 $a \leq t \leq b$
	 باشند.
	 
	 \textbf{مثال:}
	 منحنی
	 $z(t) = |t| + i \ln(1 + t) \quad  -\frac{1}{2} \leq t \leq \frac{1}{2}$
	 کمان ساده هموار است.
	 
	 \textbf{مثال:}
	 $z(t) = (t - \sin t) + i(1 - \cos t)\quad 0 \leq t \leq 2\pi$
	 مکان ساده هموار است.
	 \section{انتگرال خط}
	 فرض کنید 
	 $C$
	 یک منحنی ساده-هموار باشد که به صورت
	 $$z(t) = x(t) + i y(t) \qquad a \leq t \leq b$$
	 نمایش داده شده باشد انتگرال 
	 $f(z) = u(x, y) + i v(x, y)$
	 روی 
	 $C$
	 را با -
	 $$\int_{C} f(z) dz \quad or \quad \oint_{C} f(z) dz$$
	 نمایش داد و آن را انتگرال خطی می‌نامیم و به صورت 
	 $$\int_{C} f(z) dz = \int_{C} (u + iv)(dx + idy) = \int_{C}(u \, dx -v \, dy) + i\int (v \, dx + u \, dy)
	 $$
	 $$\text{یا}$$
	 $$\int_{C} f(z) dz = \int_{C} f(z(t)) z'(t) \, dt = \int_{a}^{b} f(x(t) + i y(t))(x'(t) i y'(t)) dt$$
	 
	 $$x^2 + y^2 = 1 \rightarrow x =\mp \sqrt{1 - y^2}$$
	 \[
	 \begin{cases}
	 	x = \cos \theta \\
	 	y = \sin \theta
	 \end{cases}
 \qquad 0 \leq \theta \leq 2\pi
	 \]
	 \[
	 \begin{cases}
	 	y = y \\
	 	x = \mp \sqrt{1 - y^2} \qquad -1 \leq y \leq 1
	 \end{cases}
	 \]
	 \textbf{نکته:}
	 اگر 
	 $\gamma_1(t) , \gamma_2(t)$
	 دو نوع نمایش پارامتری متفاوت برای فهم 
	 $C$
	 باشند آنگاه
	 $$\int_{C} f(\gamma_1(t)) \gamma_1'(t) dt = \int_{C} f(\gamma_2(t)) \gamma_2'(t) dt$$ 
	 یعنی مقدار انتگرال به نحوه پارامتری کردن خم 
	 $C$
	 بستگی ندارد.
	 
	 \textbf{مثال:}
	 $$\int_{C} f(z) dz \qquad f(z)_ = x^2 + iy^3 \qquad (0,0) \rightarrow (1, 1), \quad C: y - x^2$$
	 $$\int_{C} f(z) dz = \int_{0}{1} f(z(t)) z'(t) dt = \int_{0}^{1} = (t^2 + it^6)(1 + 2ti)dt = \int_{0}^{1} (t^2 - 2t^7 + i(2t^3 + t^6))dt$$
	 $$=\frac{1}{12} + \frac{9}{14}i$$
	 
	 \section{کاربرد انتگرال خط}
	 حاصل عبارت
	 $Re \int_{C} \overline{f(z)} dz$
	 را به عنوان مقدار کار انجام شده بوسیله نیروی
	 $\vec{f} = \vec{u} + v \vec{i}$
	 در مسیر 
	 $C$
	 تعبیر کرد که در آن 
	 $\vec{f} = u - iv$
	 مزدوج
	 $f = u + iv$
	است.
	\textbf{مثال:}
	کار انجام شده توسط نیروی
	$\vec{F} = (x+y^2) \vec{i} + x^3 \vec{j}$
	را در مسیر مستقیم از نقطه 
	$A = (0, 0)$
	تا نقطه 
	$B=(1, 2)$
	را بیابید.
	\textbf{حل:}
	معادله خط گذرنده از نقاط 
	$A$
	،
	$B$
	به صورت 
	$y = 2x$
	است.
	$$z(t) = t + 2ti \,\,;\,\, 0 \leq t\leq 1$$
	$$f(z) = x + y^2 + ix^3$$
	$$\text{کار انجام شده} = Re\int_{C} \overline{f(z)} dz = Re\int_{0}^{1} \overline{f(z(t)) z'(t)} dt = Re\int_{0}^{1} (t + 4t^3 - it^3)(1 + 2i)dt$$
	$$=Re(1 + 2i)\int_{0}^{1} (t + 4t^3 - it^3)dt =Re(1 + 2i)(\frac{1}{2} + \frac{4}{3} - \frac{1}{4} i) $$
	$$Re(1 + 2i)(\frac{11}{6} - \frac{1}{2}i) = \frac{11}{6} + \frac{1}{2} = \frac{7}{3}$$
	\newline
	\textbf{مثال:}
	نشان دهید
	\[
	\int_{C}(z - z_0)^n dz = 
	\begin{cases}
		0 \qquad n \neq -1\\
		2\pi i \qquad n = -1
	\end{cases}
	\]
	که 
	$C$
	 دایره
	 $z - z_0 = re^{i\theta}$
	 جهتی عکس عقربه های ساعت می‌باشد.
	 \textbf{حل:}
	 $\int_{C} (z-z_0)^n dz = \int_{C} (re^{i\theta})^n dz = r^n \int (e^{i\theta})^n dz = ir^{n+1} \int_{0}^{2\pi} e^{i(n+1)\theta} d\theta$
	 
	 $z = z_0 + re^{i\theta} \Rightarrow dz = ire^{i\theta}d\theta$
	 $$\int_{C} (z-z_0)^n dz = ir^{n+1} \int_{0}^{2\pi} e^{i(n+1)\theta}d\theta = ir^{n+1}(\frac{1}{i(n+1)} e^{i(n+1)\theta})_{0}^{2\pi}=r^{n+1}(\frac{1}{(n+1)} e^{i(n+1)\theta})_{0}^{2\pi}$$
	 
	 \[
	 \begin{cases}
	 	2\pi \qquad n = -1\\
	 	ir^{n+1}(\frac{1}{i(n+1)} e^{i(n+1)\theta})_{0}^{2\pi}\qquad n \neq -1
	 \end{cases}
	 \]
	 \[
	 \begin{cases}
	 	2\pi i \qquad n = -1\\
	 	0   \qquad n\neq -1
	 \end{cases}
	 \]
	 $$e^{i(n+1)\theta} \big|_{0}^{2\pi} = \cos(2\pi(n+1)) + i\sin(2\pi(n+1)) = -e^0 = 0$$
	 
	 \section{دامنه همبند ساده}
	 دامنه
	 $D$
	 را همبند ساده می‌گوییم هرگاه درون خم ساده بسته واقع در 
	 $D$
	 تماما در
	 $D$
	 باشد.
	 $$D= \{x + iy|x, y \in \mathbb{Q}\}$$
	 \newline
	 \textbf{قضیه کشی-گورسا}
	 اگر تابع
	 $f(z)$
	 در درون و روی مرز ساده بسته
	 $C$
	 تحلیلی باشد آنگاه
	 $$\oint_{C} f(z) dz = 0$$
	 
	 \textbf{مثال:}
	 $$\int_{|z| =1} \frac{e^z}{\cos z} = 0$$
	 
	 \textbf{نکته:}
	 فرض کنید
	 $f(z)$
	 بر مرز 
	 $C$
	 که به درازای 
	 $L$
	 است پیوسته باشد و بر 
	 $C$
	 $|f(z)| < M$
	 آنگاه 
	 $$|\int_{C} f(z)dz| \leq \int_{C} |f(z) |dz| \leq M \qquad |dz| = ML$$
	 
	 \section{فرمول انتگرال کشی}
	 فرض کنید تابع
	 $f(z)$
	 در دامنه همبند ساده که شامل مرز ساده 
	 $C$
	 است تحلیلی باشد اگر 
	 $z_0$
	 درون 
	 $C$
	 باشد آنگاه
	 $$f(z_0) = \frac{1}{2\pi i} \int_{C} \frac{f(z)}{z - z_0} dz \Rightarrow \int_{C} \frac{f(z)}{z - z_0} = 2\pi i f(z_0)$$
	 
	 \textbf]{مثال:}
	 انتگرال زیر را محاسبه کنید.
	 $$\int_{|z| =1} \frac{\cosh z}{z^2 - 2z}\, dz$$
	 
	 $$\oint_{|z| = 1} \frac{\frac{\cosh z}{z - 2}}{z - 0} dz = 2\pi f(0) = 2\pi i \times (-\frac{1}{2}) = -\pi i$$
	 $$f(z) = \frac{\cosh }{ z -2} \Rightarrow f(0) = \frac{1}{-2}$$
	 
	 \section{تعمیم فرمول انتگرال کشی}
	 فرض کنید تابع
	$f(0)$
	در دامنه همبند ساده که شامل مرز ساده 
	$c$
	است، تحلیلی باشد اگر 
	$z_0$
	درون 
	$c$
	باشد آنگاه 
	$f(z)$
	در نقطه 
	$z_0$
	دارای مشتق از هر مرتبه ای است که تحلیلی اند و از رابطه زیر محاسبه می‌شود
	$$\int_{C} \frac{f(z)}{(z - z_0)^{n+1}} dz = 2\pi i \frac{f^{(n)}(z_0)}{n!}$$
	\textbf{مثال:}
	اگر
	$c$
	دایره واحد
	$|z| = 1$
	باشد در جهت مثبت باشد و
	$f(z) = \exp(2z)$
	آنگاه
	$\int_{C} \frac{\exp(2z)}{z^4} dz$
	را محاسبه کنید.
	$$\int_{C} \frac{\exp(2z)}{z^4} dz = \frac{2\pi i }{3!} f^{(3)}(0) = \frac{2\pi i }{3!} \times 8$$
	$$f(z) = \exp(2z) \qquad
	f'(z) = 2\exp(2z)\qquad
	$$
	$$f''(z) = 4\exp(2z)\qquad
	f'''(z) = 8\exp(2z) \Rightarrow f'''(0) = 8$$
	\newline
	\textbf{مثال:}
	حاصل انتگرال زیر را محاسبه کنید.
	$$\int_{|z| =\frac{7}{2}} \frac{e^z}{(z - 3)(z+1)^2}\,dz$$
	نقاط غیر تحلیلی
	$z = -1, 3$
	در داخل
	$|z| =\frac{7}{2}$
	قرار دارند.
	$$\int_{C} \frac{e^z}{(z - 3)(z+1)^2}\, dz = \int_{C_1} \frac{e^z}{(z - 3)(z+1)^2}\, dz + \int_{C_2} \frac{e^z}{(z - 3)(z+1)^2}\, dz$$
	$$\int \frac{e^z}{(z - 3)(z+1)^2}\, dz = \int_{C_1} \frac{\frac{e^z}{z - 3}}{(z+1)^2}\, dz +\int_{C_2} \frac{\frac{e^z}{(z+ 1)^2}}{(z-3)}\, dz = \frac{2\pi i}{1!} f'(-1) + 2\pi i g(3)$$
	$$f(z) = \frac{e^z}{z - 3} \Rightarrow f'(z) = \frac{e^z(z - 3) - e^z}{(z - 3)^2} \Rightarrow f'(-1) = \frac{\frac{1}{e}(-4) - \frac{1}{e}}{16} = -\frac{5}{16e}$$
	$$g(z) = \frac{e^z}{(z + 1)^2} \Rightarrow g(3) = \frac{e^3}{16}$$
	$$\int_{C} \frac{e^z}{(z - 3)(z+1)^2}\, dz = 2\pi i(-\frac{5}{16e} + \frac{e^3}{16})$$
	
	\textbf{قضیه  مررآ(عکس قضیه کشی-گورسا)}
	
	فرض کنید
	$f'(z)$
	در دامنه 
	$D$
	پیوسته باشد اگر بر هر مرز ساده بسته 
	$c$
	درون 
	$D$
	،
	$\oint_{C} f(z) dz = 0$
	،
	$f(z)$
	در 
	$D$
	تحلیلی است.
	
	\section{نامساوی کشی}
	فرض کنید
	$C$
	دایره 
	$|z - z_0| = r$
	و 
	$M$
	یک کران بالای 
	$|f(z)|$
	یعنی 
	$|f(z)| \leq M$
	$$|f^{(n)}(z_0) \leq \frac{Mn!}{r^n} \quad ;\quad n = 0, 1,2 \dots $$
	
	\textbf{اثبات:}
	$$f^{(n)}(z_0) =\frac{n!}{2\pi i} \oint_{C} \frac{f(z)}{(z - z_0)^{n+1}}dz \Rightarrow |f^{(n)}(z_0)| =\frac{n!}{2\pi} |\frac{f(z)}{(z - z_0)^{n+1}}dz|$$
	$$\leq\frac{n!}{2\pi} \frac{|f(z)|}{|(z - z_0)^{n+1}|}dz \leq \frac{n!}{2\pi} \frac{M}{r^{n+1}}\oint dz = \frac{n!}{2\pi} \frac{M}{r^{n+1}} 2\pi(r) = n! \times \frac{M}{r^n}$$
	\newline
	\textbf{قضیه لیوویل:}
	اگر تابع
	$f(z)$
	تحلیلی و 
	$|f(z)| \leq M$
	آن‌گاه تابع 
	$f(z)$
	تابع ثابت است.
	\textbf{قصیه معادله چندجمله ای}
	$$a_nz^n + a_{n -1}z ^{n-1} + \dots + a_1z + a_0 = 0 \,\,, \,\,a_n \neq 0$$
	دقیقا 
	$n$
	ریشه دارد در 
	$\mathbb{C}$
	\newline
	\textbf{قضیه مقدار میانگین گاس:}
	اگر 
	$f(z)$
	در داخل و روی مرز دایره
	$C :|z - a| = r$
	تحلیلی باشد آن‌گاه
	$f(a)$
	میانگین مقدار 
	$f(z)$
	روی 
	$C$
	خواهد بود یعنی
	$$2\pi \times f(a) = \int_{0}^{2\pi} f(a + re^{i\theta})d\theta$$
	\newline
	\textbf{مثال:}
	انتگرال زیر را محاسبه کنید.
	$$\int_{0}^{2\pi} \cos^2(\frac{\pi}{6} + 2e^{i\theta})d\theta=\frac{3}{4} \times 2\pi = \frac{3\pi}{2}$$
	$$f(z) = \cos^2(z) \Rightarrow \cos^2(\frac{\pi}{6}) = \frac{3}{4}$$
	
	\textbf{قضیه ماکزیمم توابع}
	
	اگر تابع
	$f(z)$
	را در دامنه کراندار
	$D$
	تحلیلی و غیر ثابت باشد و بر بستار آن یعنی
	$\overline{D}$
	پیوسته باشد آن‌گاه
	$|f(z)|$
	مقدار ماکزیمم خود را روی مرز اختیار می‌کند.
	\newline
	\textbf{قضیه مینیمم تابع}
	
	اگر تابع 
	$f(z)$
	در دامنه کراندار
	$D$
	تحلیلی و بر 
	$z \in D$
	داشته باشیم
	$f(z) \neq 0$
	همچنین بستار آن یعنی
	$\overline{D}$
	پیوسته و 
	$\overline{D}$
	فشرده باشد آن‌گاه
	$|f(z)|$
	مقدار مینیمم خود را در مرز اختیار می‌کند.
	\newline
	\textbf{مثال:}
	مقدار ماکزیمم و مینیمم 
	$|f(z)|$
	با تعریف 
	$f(z) = e^z$
	را روی 
	$|z| = r$
	با 
	$r > 0$
	پیدا کنید.
	
	طبق قضیه 
	$\min$
	و 
	$\max$
	توابع مختلط
	$\min$
	و 
	$\max$
	بر نقاط مرزی دامنه اتفاق می‌افتد.
	$$|e^z| = |e^{x + iy}| = |e^x||e^{iy}| = e^x|\cos y + i\sin y| = e^x$$
	تابع 
	$e^x$
	بر 
	$[-r, r]$
	صعودی است لذا
	$e^{-r} < e^r$
	پس
	$$\min_{|z| = r} |e^z| = \min e^x = e^{-r}$$
	$$\max_{|z| = r} |e^z| = \max e^x = e^{r}$$
	------------------- امتحان میان تریم تا اینجا------------------------------------------
	\chapter{دنباله و سری های مختلط}
	
	\textbf{تعریف دنباله:}
	تابعی مختلط است که دامنه آن اعداد مختلط طبیعی و بر آن مجموعه غیر تهی مختلط است
	$$ \underset{n \to f(n) = z_n}{f: \mathbb{N} \to \mathbb{C}}$$
	مقادیر برد را با 
	$z_n$
	یا به طور خلاصه 
	$\{z_n\}_{n =1}^{+\infty}$
	نمایش می‌دهیم.
	
	دنباله 
	$\{z_n\}$
	را همگرا می‌گوییم هرگاه
	$$\lim_{n \to +\infty} z_n = C$$
	$$\forall \epsilon > 0 \quad\exists N \forall n(n > N \Rightarrow |z_n - c| < \epsilon)$$
	\newline
	
	\textbf{قضیه:}
	دنباله‌ی
	$\{z_n\} = \{x_n + i y_n\}_{n  = 1}^{\infty}$
	را همگرا به 
	$z = a + ib$
	گوییم هرگاه دنباله های متشکل از قسمت های حقیقی
	$\{x_n\}$
	و
	$\{y_n\}$
	موهومی به ترتیب به
	$a$
	و
	$b$
	همگرا باشند
	\newline
	
	\textbf{مثال:}
	همگرایی دنباله
	$z_n = s : n (\frac{1}{n}) + i(-1)^n$
	را بررسی کنید.
	
	همگرا نیست زیرا
	$(-1)^n$
	در 
	$n \to \infty$
	همگرا نسیت.
	\newline
	
	\textbf{مثال:}
	همگرایی دنباله 
	$z_n = \frac{1}{n^3} + i$
	را بررسی کنید.
	$$\lim_{z \to \infty} \frac{1}{n^3} = 0 \quad , \lim_{z \to \infty} 1 = 1 \Rightarrow z_n \to i$$
	\newline
	
	\textbf{تعریف:}
	سری نامتناهی
	$\Sigma z_n = z_1 + z_2 +\dots + z_n + \dots$
	از اعداد مختلط است اگر دنباله‌ی
	$s_n = \Sigma_{k = 1} z_k =z_1 + z_2 +\dots + z_n $
	از مجموع های جزئی به 
	$s$
	همگرا باشد. در این صورت می‌نویسیم
	$$\Sigma_{k = 1}^{\infty} z_n = s$$
	\newline
	
	\textbf{مثال:}
	سری هندسی 
	$\Sigma z^n = 1 + z^1 + z^2 + \dots + z^n + \dots$
	را بدست آورید.
	$$\Sigma_{n = 0}^{\infty} z^n = \frac{1}{1  -z} \qquad (|z| < 1)$$
	$$\Sigma_{n = 0}^{\infty} (-z)^n  = \frac{1}{1  +z} \qquad (|z| < 1)$$
	\newline
	
	\textbf{مثال:}
	$$\Sigma_{n = 0}^{\infty} z^{2n} = ? \Rightarrow \Sigma_{n = 0}^{\infty} z^{2n} = \frac{1}{1 - z^2} \quad (|z| < 1 \, or \, |z^2| < 1)$$
	\newline
	
	\textbf{دنباله توابع:}
	فرض کنید
	$f_1(z), f_2(z) , \dots$
	که به طور خلاصه به صورت
	$f(z)$
	نمایش داده می‌شوند دنباله‌ای از توابع تک مقداری تعریف شده بر حسب 
	$z$
	در ناحیه مشخص از صفحه‌ی مختلط باشد.
	\newline
	
	\textbf{تعریف:}
	تابع
	$F(z)$
	واحد
	$\{f_n\}$
	می‌نامیم و می‌نویسیم
	$\lim_{n \to \infty} f_n(z) = F(z)$
	هرگاه
	$$pointwise: \forall \epsilon > 0 \quad \exists N(\epsilon, z) \,;\, \forall n(n > N(\epsilon, z)\Rightarrow |f_n(z) - F(z)| < \epsilon)$$
	در بعضی مقدار 
	$N$
	با تغییر
	$\epsilon$
	و
	$z$
	تغییر خواهد کرد.
	\newline
	
	\textbf{تعریف:}
	همگرایی یکنواخت: دنباله توابع
	$\{f_n(z)\}$
	به 
	$f(z)$
	به طور یکنواخت همگراست هرگاه
	
	$$uniformly  : \forall \epsilon > 0 \quad \exists N(\epsilon) \, ;\, \forall z \,,\,\forall n (n > N(\epsilon) \Rightarrow |f_n(z) - f(z)| < \epsilon)$$
	\newline
	
	\textbf{مثال:}
	$$f_n(z) = \frac{1}{nz} \quad z \neq (0, 0)$$
	$$f_n(z) \overset{P.W}{\rightarrow} 0 \quad |z|<1$$
	طبق خاصیت ارشمیدسی
	$\exists N$
	بطوریکه
	$\frac{1}{N} < \epsilon$
	برای هر 
	$n \geq N$
	و هر
	$|z|> 1$
	داریم:
	$$|f_n(z) - f(z)| = |\frac{1}{nz} - 0| = |\frac{1}{nz}| = \frac{1}{n}\frac{1}{|z|} \leq \frac{1}{n} \leq \frac{1}{N} < \epsilon$$
	\newline
	
	\textbf{قضیه:}
	اگر
	$\Sigma_{n = 0}^{\infty}M_n$
	همگرا باشد آن‌گاه
	$\Sigma_{n = 0}^{\infty}f_n(z)$
	همگرای یکن.اخت روی
	$D$
	است.
	\newline
	
	\textbf{قضیه:}
	اگر
	$\Sigma_{n = 0}^{\infty}f_n(z)$
	در ناحیه 
	$D$
همگرای یکینواخت به تابع
$f(z)$
باشد در این ناحیه تمام توابع
$f_n(z)$
پیوسته باشند و منحنی 
$c$
در 
$D$
واقع باشد آن‌گاه داریم:
$$\int_{c} \Sigma_{n = 0}^{\infty}f_n(z) dz = \Sigma_{n = 0}^{\infty} \int\}_{n  = 1}  f_n(z) dz \quad (z \in D)$$
\newline

\textbf{مثال:}
	با انتگرال گیری از سری زیر روی منحنی 
	$c$
	از
	$t = 0$
	تا
	$t = z$
	نتیجه را بیان کنید.
	$$ \Sigma_{n = 0}^{\infty} t^n = \frac{1}{1 - t} \quad |t| < 1$$
	
	سمت چپ:
	$$\int_{0}^{z} \frac{1}{1 - t} dt=\ln (1 - t) |_{0}^{z} = - \ln (1 - z)$$
	\newline
	
	سمت راست:
	$$\int \Sigma_{n = 0}^{\infty} t^n dt = \Sigma_{n = 0}^{\infty} \int_{0}^{z} t^n dt = \Sigma_{n = 0}^{\infty} (\frac{t^{n + 1}}{n+1})|_{0}^{z} =\Sigma_{n = 0}^{\infty} (\frac{z^{n + 1}}{n+1}) = \Sigma_{n = 0}^{\infty} \frac{z^n}{n} $$
	\newline
	
	
	
	
	
	
	
	
	
	
	
	
	
	
	
	
	
	
	
	
	
	
	\textbf{قضیه:}
	اگر 
	$\Sigma_{n = 1}^{+\infty} f_n(z)$
	در ناحیه 
	$\mathbb{R}$
	همگرای یکنواخت تابع 
	$f(z)$
	باشد و در این ناحیه تمام توابع
	$f_n(z)$
	تحلیلی باشند. آنگاه در هر نقطه داخل
	$z \in \mathbb{R}$
	،
	داریم:
	$$\frac{df(z)}{dz} = \Sigma_{n = 1}^{+\infty}\frac{df_n(z)}{dz} ; z \in \mathbb{R}$$
	
	\textbf{مثال:}
	با مشتق گیری از سری 
	$\Sigma_{n = 1}^{+\infty} z^n = \frac{1}{1 - z}$
	در دامنه 
	$|z|<1$
	نتیجه زا بیان کنید.
	
	$$\frac{d}{dz} \Sigma_{n = 1}^{+\infty} n z^{n - 1} ; |z|< 1$$
	$$\frac{d}{dz} \frac{1}{1 - z} = \frac{1}{(1 - z)^2} \Rightarrow \Sigma_{n = 1}^{+\infty} n z^{n - 1} = \frac{1}{(1 - z)^2} \qquad |z|<1$$
	\newline
	
	\textbf{قضیه:}
	آزمون های همگرایی سری های مختلط
	\begin{enumerate}
		\item 
		آزمون نسبت اگر 
		$\lim_{n \to +\infty} |\frac{f_{n +1} (z)}{f_n(z)}| = L$
		
		اگر 
		$L < 1$
		,آن‌گاه
		$\Sigma_{n = 1}^{+\infty} f_n(z)$
		همگراست
		
		اگر 
		$L>1$
		آن‌گاه سری فوق واگراست.
		\item
		آزمون ریشه.اگر
		$$\lim_{n \to +\infty} |f_n(z)|^{\frac{1}{n}}= L$$
		اگر
		$L<1$
		سری فوق همگراست.
		
		اگر
		$L < 1$
		سری واگراست.
		\item
		آزمون مقایسه اگر برای 
		$n > N$
		داشته باشیم
		$|g_n(z)| > |f_n(z)|$
		و
		$\Sigma_{n = 1}^{+\infty} g_n(z)$
		همگرا باشد آن‌گاه
		$\Sigma_{n = 1}^{+\infty} f_n(z)$
		همگراست و اگر
		$\Sigma_{n = 1}^{+\infty} g_n(z)$
		واگرا شود آنگاه
		$\Sigma_{n = 1}^{+\infty} g_n(z)$
		واگراست.
		\item
		آزمون رابه :‌اگر داشته باشیم 
		$$\lim_{n \to +\infty} n |1  - \frac{f_{n +1} (z)}{f_n(z)}| = L$$
		اگر
		$L<1$
		آن‌گاه سری
		$\Sigma_{n = 1}^{+\infty}$
		واگراست
	
		اگر 
		$L > 1$
		سری فوق همگراست.

	\end{enumerate}
	

\textbf{مثال:}
ناحیه همگرایی
$\Sigma_{n = 0}^{+\infty} \frac{z^n}{n!}$
را تعیین کنید.
$$\lim_{n \to +\infty} |\frac{f_{n +1} (z)}{f_n(z)}| = \lim_{n \to +\infty} |\frac{\frac{z^{n +1}}{(n+ 1)!}}{\frac{z^n}{n!}}| = \lim_{n \to +\infty} |\frac{z}{n+1}| = 0 < 1$$ 
پس ناجیه همگرایی صفجه مختلط است.
\newline

\textbf{مثال:}
ناحیه همگرایی
$\Sigma_{n = 0}^{+\infty} ne^{-nz^2}$
را تعیین کنید.
$$\lim_{n \to +\infty} (|ne^{-nz^2}|)^{\frac{1}{n}} = \lim_{n \to +\infty} n^{\frac{1}{n}}|e^{-nz^2}|^{\frac{1}{n}} = \lim_{n \to +\infty} |e^{-z^2}| = |e^{-z^2}| = |e^{-(x + iy)^2}| = e^{y^2 - x^2} < 1$$




\section{سری توانی}
یک سری به فرم 
$\Sigma_{n = 0}^{+\infty} a_n(z - z_0)^n$
سری توانی
$zz_0$
نامیده می‌شود.
با استفاده از آزمون نسبت تعریف می‌کنیم
$$R = \lim_{n \to +\infty} |\frac{a_n}{a_{n+1}}|$$
که 
$R$
شعاع همگرایی نامیده می‌شود.
در این صورت اگر
$|z - z_0|<R$
آن‌گاه سری توانی همگرای یکنواخت است و اگر 
$|z - z_0|>R$
آن‌گاه سری واگراست و برای
$|z - z_0| = R$
باید بررسی شود و آزمون نسبت جواب نمی‌دهد

\section{سری تیلور}
فرض کنید
$f(z)$
در دامنه 
$D$
تحلیلی و مرز
$D$
،
$c$
است.
اگر
$z_0 \in D$
آن‌گاه نمایش دیگری برای
$f(z)$
به صورت زیر می‌باشد.
$$f(z) = \Sigma_{n = 0}^{+\infty} a_n(z - z_0)^n$$
$$a_n = \frac{1}{2\pi i} \int_{c} \frac{f(z)}{(z - z_0)^{n + 1}}dz = \frac{2\pi i }{2\pi i} \frac{f^(n)(z_0)}{n!} = \frac{f^(n)(z_0)}{n!} \quad n = 1, 2, \dots$$
\newline

\textbf{مثال:}
تابع
$f(z) = \frac{1}{1 - z}$
برای
$z \neq 1$
نمایش سری تیلور حول
$z = z_0 \neq 1$
به صورت زیر است
$$f(z) = \Sigma_{n = 0}^{+\infty} \frac{f^{(2n)}(z_0)}{n!}(z - z_0)^n$$
برای وقتی که 
$|z - z_0| < |1 - z_0|$
برقرار است.
\section{سری مکلورن}
اگر مقدار
$z_0$
در سری تیلور برابر صفر باشد، سری را مکلورن می‌نامیم.
\section{نمایش سری مکلورن توابع خاص}
\begin{enumerate}
	
	\item
	$$e^z = 1 +z + \frac{z^2}{2!} + \dots = \Sigma_{n = 0}^{\infty} \frac{z^n}{n!} \quad |z|< \infty$$
	\item
	$$\sin z = z - \frac{z^3}{3!} +  \frac{z^5}{5!} - \dots = \Sigma_{n = 0}^{\infty} (-1)^{n -1 } = \frac{z^{2n - 1}}{(2n - 1)!}\qquad |z| < \infty$$
	\item
	$$\cos z = 1 - \frac{z^2}{2!} +  \frac{z^4}{4!} - \dots = \Sigma_{n = 0}^{\infty} (-1)^{n } = \frac{z^{2n }}{(2n)!}\qquad |z| < \infty$$
	\item
	$$\ln (1 + z) = z - \frac{z^2}{2} + \frac{z^3}{3} - \frac{z^4}{4} + \dots = \Sigma_{n = 0}^{\infty} (-1)^{n +1} = \frac{z^{n }}{n}\qquad |z| < 1$$
	
	
\end{enumerate}
\section{تکین ها}
نقطه
$z_0$
را تکین تابع
$f(z)$
گوییم هرگاه در 
$z_0$
تحلیلی باشد و در نقطه‌ای از همسایگی
$z_0$
تحلیلی باشد.
\newline

\textbf{مثال:}
$z = 0$
برای تابع
$f(z) = \frac{1}{z}$
تکین است.









\section{تکین تنها}
نقطه
$z_0$
را تکین تنهای تابع 
$f(z)$
گوییم هرگاه
$z_0$
تکین باشد و همسایگی محذوفی از
$z_0$
وجود داشته باشد به طوریکه
$f(z)$
در سرتاسر این همسایگی تحلیلی باشد.
\section{تکین غیر تنها}
نقطه
$z_0$
را تکین تنهای تابع
$f(z)$
گوییم هرگاه هر همسایگی 
$z_0$
شامل نقطه تکین دیگری غیر از 
$z_0$
باشد.
\section{تکین غیر تنها از نوع انباشتگی}
نقطه 
$z_0$
را تکین غیر تنها از نوع انباشتگی تابع 
$f(z)$
گوییم هرگاه دنباله ای از نقاط تکین مانند
$\{z_n\}$
وجود داشته باشد به طوریکه 
$\lim_{n \to \infty} z_n = z_0$
\newline

\textbf{مثال:}
تابع
 $\frac{1}{z}$
 نقطه تکین 
 $z = 0$
 که از نوع نکین تنهاست.

\textbf{مثال:}
$$f(z) = \frac{z + 1}{z^3(z^2 + 1)} \quad z =0 , z = i, z = -i$$


\textbf{مثال:}
$$f(z) = \frac{1}{\sin (\frac{\pi}{z})} e^{(\tan \frac{1}{z})}$$
$$z  = \frac{1}{n} \quad ; \quad z = 0 \quad n \in \mathbb{N}$$
$z = 0$
نقطه تکین غیر تنهاست
$$\forall r > 0 \quad ; \quad 0<|z|< r$$
$$\exists n_0 \in \mathbb{N} \quad ;\quad \frac{1}{n_0} < r$$
و از طرفی
$\frac{1}{n_0}$
نقطه تکین تنها متمایز از صفر است پس 
$z = 0$
تکین غیر تنها است.
از طرفی نقاط
$z = \frac{1}{n}$
که 
$n \in \mathbb{N}$
است نقاط تکین تنها می‌باشند.
\section{سری لوران}
اگر تابع
$f(z)$
در نقطه 
$z = z_0$
دارای تکین تنها باشد آن‌گاه در همسایگی 
$z_0$
،
تابع دارای سری لوران است فرض کنید
$f(z)$
در داخل طوقه 
$r_1<|z- z_0| < r_2$
تحلیلی باشد در این صورت نمایش 
$f(z)$
به صورت زیر است


$$f(z) = \Sigma_{n = 0}^{\infty} a_n(z- z_0)^n(\text{قسمت تحلیلی}) +\Sigma_{n = 1}^{\infty} b_n(z- z_0)^{-n} = \Sigma_{-\infty}^{\infty} c_n‌(z- z_0)^{n}$$
\newline

\textbf{مثال:}
تابع:
$$a_n = \frac{1}{2\pi i} \int_{c} \frac{f(z)}{(z_z_0)^{n+1}}$$
$$b_n = \frac{1}{2\pi i} \int_{c} \frac{f(z)}{(z_z_0)^{1 - n}}$$
$$c_n = \frac{1}{2\pi i} \int_{c} \frac{f(z)}{(z_z_0)^{n+1}}$$
\newline

\textbf{نکته:}
$$\oint_{c} f(z) dz = \int \Sigma_{n = 0}^{\infty} a_n(z- z_0)^n dz + \int \Sigma_{n = 1}^{\infty} b_n(z- z_0)^{-n} dz = 2\pi i b_1$$
$$Res f(z) = b_1 \quad z = 0 \quad$$
$b_1$
را مانده 
$f(z)$
در
$z = z_0$
می‌نامیم.
\newline

\textbf{مثال:}
سری لوران تابع 
$f(z) = \frac{1}{z(z - 1) (z- 2)}$
حول 
$z = 0$
در دامنه 
$1 < |z|< 2$
بنویسید.
$$f(z) = \frac{1}{2z} + \frac{1}{1 - z} + \frac{1}{2(z - 2)}$$
$$\frac{1}{1 - z} = - \frac{1}{z} \frac{1}{1 - \frac{1}{z}} = -\frac{1}{z} = \Sigma_{n = 0}^{\infty} 
\frac{1}{z^n}$$
$$\frac{1}{2(z - 2)} = -\frac{1}{4} \frac{1}{1 - \frac{z}{2}} = - \frac{1}{4} \Sigma_{n = 0}^{\infty} (\frac{z}{2})^n$$
در نهایت
$$f(z) = \frac{1}{2z}  -\frac{1}{z} \frac{1}{1 - \frac{1}{z}}-\frac{1}{4} \frac{1}{1 - \frac{z}{2}} = \frac{1}{2z} - \frac{1}{z} \Sigma_{n = 0}^{\infty} \frac{1}{z^n} - \frac{1}{4}\Sigma_{n = 0}^{\infty} (\frac{z}{2})^n$$
\newline

\textbf{مثال:}
تابع
$f(z) = \frac{1}{(z+ 1)(z+ 3)}$
را به یک سری لوران در ناحیه 
$0< |z+ 1|< 2$
بسط دهید.
\textbf{حل:}
فرض کنید
$z + 1 = u$
;
$0<|u|<2$
$$f(z) = \frac{1}{(z+ 1)(z+3)} = \frac{1}{u(u + 2)} = \frac{1}{2u(\frac{u}{2} + 1)} = \frac{1}{2u} \Sigma_{n = 0}^{\infty} (-1)^n(\frac{u}{2})^n$$
\textbf{نکته:}
چون
$|u|< 2$
پس
$|\frac{u}{2}|< 1$
پس سری هندسی زسر همگراست.
$$\frac{1}{1 + \frac{u}{2}} = \Sigma_{n = 0}^{\infty} (-1)^n(\frac{u}{2})^n$$
پس
$$f(z) = \frac{1}{2u}\Sigma_{n = 0}(-1)^n(\frac{u}{2})^n = \frac{1}{2u} + \Sigma_{n = 0}(-1)^n(\frac{u^{n - 1}}{2^{n+1}}) = $$
$$\frac{1}{2(z+ 1)} + \Sigma_{n = 0}^{\infty} (-1)^n \frac{(z+ 1)^{n - 1}}{2^{n + 1}}$$

\section{انواع نقاط تکین تنها}
\begin{enumerate}
	\item
	نقطه تکین برداشتنی: اگر بخش اصلی بسط سری لوران تابع
	$f(z)$
	برابر باشد، تکین را برداشتنی گوییم.
	\item
	قطب: اگر بخش اصلی بسط سری لوران تابع
	$f(z)$
	از تعداد نامتناهی جمله تشکیل شده باشد آن‌گاه تکین را قطب می‌نامیم. توان آخرین جمله 
	(مثلا
	$m$
	)
	را مرتبه قطب می‌نامیم اگر 
	$m = 1$
	آن‌گاه قطب را ساده می‌گوییم.
	$$f(z) = \Sigma_{n = 0}^{+\infty} a_n(z - z_0)^n + \Sigma_{n = 1}^{m} b_n (z - z_0)^{-n}$$
	$$(z - z_0)^m f(z) = \Sigma_{n = 0}^{\infty} a_n(z - z_0)^{n+ m} + \Sigma_{n = 1}^{m} b_n (z - z_0)^{m-n}$$
	با 
	$m - 1$
	مشتق گیری از طرفین و سپس بررسی حد در طرف تساوی وقتی
	$z \to z_0$
	خواهیم داشت:
	$$b_1 = \frac{1}{(m - 1)!} \lim_{z \to z_0} \frac{d^{m - 1}}{d z^{m - 1}}(z - z_0)^m f(z) = Res_{z = z_0}(f(z))$$
	
	\item
	تکین اساسی: اگر بخش اصلی بسط سری لوران تابع
	$f(z)$
	دارای  تعداد نامتناهی جمله باشد تکین را اساسی گوییم.
	
	
\end{enumerate}
\newline

\textbf{مثال}
\begin{enumerate}
	\item 
	تابع
	$$f(z) = \frac{\sin z}{z} = \frac{1}{z}(z - \frac{1}{3!} z^3 + \frac{z^5}{5!} + \dots) = (1 -\frac{1}{3!} z^2 + \frac{z^4}{5!} + \dots)$$
	پس 
	$z = 0$
	تکین تنها از نوع برداشتنی است.
	\item
	تابع
	$$f(z) = \frac{\sin z}{z^2} = \frac{1}{z^2}(z - \frac{1}{3!} z^3 + \frac{z^5}{5!} + \dots) = (\frac{\sin z}{z} -\frac{1}{3!} z + \frac{z^3}{5!} + \dots)$$
	\item
	تابع
	$$e^{\frac{1}{z}}$$
	
	
\end{enumerate}
\end{document}
